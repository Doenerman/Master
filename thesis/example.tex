\documentclass[10pt]{article}
\usepackage{geometry}                % See geometry.pdf to learn the layout options. There are lots.
\usepackage{blindtext}
\usepackage[parfill]{parskip}    % Activate to begin paragraphs with an empty line rather than an indent
\usepackage{tikz}
\usetikzlibrary{arrows,automata,shadows,positioning,shapes}
\usepackage{graphicx}
\usepackage{amssymb}
\usepackage{amsmath}
\usepackage{amsthm}
\usepackage{epstopdf}
\usepackage{hyperref}
\usepackage{listings}
\usepackage{subfiles}
\usepackage[utf8]{inputenc}
\usepackage{float}
\usepackage{tikz}
\usepackage{graphicx}
\usepackage{caption}
\usepackage{wrapfig}

\theoremstyle{definition}
\newtheorem{definition}{Definition}
\theoremstyle{plain}
\newtheorem{lemma}{Lemma}
\newtheorem{theorem}{Theorem}
\theoremstyle{remark}
\newtheorem{remark}{Remark}
\newtheorem{example}{Example}


%variables
\newcommand{\verDistVert}{3}
\newcommand{\verDistHorz}{3}

\begin{document}

\begin{example}
  \begin{displaymath}
    L=\{(a^{i_j}b)^\omega|\forall j: i_j \text{devidable by }j=\Pi\limits_{i^\prime,\dots,m\}p_{i^\prime},\text{ where $p_{i^\prime}$ are prime numbers\}
  \end{displaymath}
  \begin{figure}
    \begin{tikzpicture}
      \node[state,initial]  at (0,0)     {$3$};
      \node[state]          at (\veristVert,0) {$2$};
      \node[state]          at (\verDistVert,\verDistHorz) {$3$};
      \node[state]          at (\verDistVert,2*\verDistHorz) {$3$};
      \node[state]          at (0,\verDistHorz) {$1$};
      \node[state]          at (0,\verDistHorz) {$3$};
      \node[state]          at (-\verDistVert,0) {$3$};
      \node[state]          at (-\verDistVert,\verDistHorz) {$3$};
    \end{tikzpicture}
  \end{figure}
\end{example}
  
\end{document}
