\documentclass[10pt]{article}
\usepackage{geometry}                % See geometry.pdf to learn the layout options. There are lots.
\usepackage{blindtext}
\usepackage[parfill]{parskip}    % Activate to begin paragraphs with an empty line rather than an indent
\usepackage{tikz}
\usetikzlibrary{arrows,automata,shadows,positioning,shapes}
\usepackage{graphicx}
\usepackage{amssymb}
\usepackage{amsmath}
\usepackage{amsthm}
\usepackage{epstopdf}
\usepackage{hyperref}
\usepackage{listings}
\usepackage{subfiles}
\usepackage[utf8]{inputenc}
\usepackage{float}
\usepackage{tikz}
\usepackage{graphicx}
\usepackage{caption}
\usepackage{wrapfig}

\theoremstyle{definition}
\newtheorem{definition}{Definition}
\theoremstyle{plain}
\newtheorem{lemma}{Lemma}
\newtheorem{theorem}{Theorem}
\theoremstyle{remark}
\newtheorem{remark}{Remark}

\begin{document}

\begin{lemma}
  Let $\mathfrak{A}=(Q,\Sigma,q_0,\delta,c),c:Q\rightarrow \{0,\dots,k\}$ be a DPA and $(\mathfrak{A}_0,\dots,\mathfrak{A}_k)$ be its DFA representation. Let $l\subseteq Q$ be a set of states such that for all $q\in l$ there exists a word $w\in\Sigma^+$ and $\delta^+(q,w)=q$. 
  If two states in $q,q^\prime \in l$ are merged in all DFA $\mathfrak{A}_i$ and $w \in \Sigma^+$ is the shortest word, such that $\delta(q,w)=q^\prime$ holds and $c(q)=i$ the following must be true:
  \begin{enumerate}
    \item $|l| \mod |w| = 0$
    \item every $|w|$-th state of $l$ must have the color $i$
  \end{enumerate}
  \begin{proof}
  If $q$ and $q^\prime$ are merged and the previous mentioned does not hold, there would either be a word $w\in\Sigma^\star$ with $|w| \mod |w^\prime| \neq 0$ and $\delta^\star(q,w)=q$ does not hold, or not every $|w|$-th state in $l$ does not have color $i$. If $|w| \mod |w^\prime| \neq 0$ hold, there exists a state $q_p\in l$ and a word $w\in\Sigma^+$ with $c(q_p)=c(\delta^+(q_p,w))$, in particular this holds for $\delta^+(q_p,w)=q_p$, but not $q_p\in F_i \Leftrightarrow \delta^+_i(q_p,w)=q$. If not every $|w|$-th state in $l$ have color $i$ but $q$ and $q^\prime$ are merged, then there is a word there is a word $w\in\Sigma^+$ such that $c(q)=c(\delta^+(q,w))$ and $\delta^+(q,w)=q^\prime$ and $q_s\in F_i$ but $\delta_i(q_s,w)\not\in F_i$. One fitting word would be the word that visits every node in $l$ exactly once.
  \end{proof}
\end{lemma}

\begin{lemma}
  Let $\mathfrak{A}=(Q,\Sigma,q_0,\delta,c),c:Q\rightarrow \{0,\dots,k\}$ be a DPA and $(\mathfrak{A}_0,\dots,\mathfrak{A}_k)$ be its DFA representation. Let $l\subseteq Q$ be a set of states such that for all $q\in l$ there exists a word $w\in\Sigma^+$ and $\delta^+(q,w)=q$.
  If the DPA $\mathfrak{A}$ is minimal, then each loop $l\subseteq Q$ there exists an automaton $\mathfrak{A}_i$ that has a loop $l_i^\prime\subseteq Q_i$ of the same size as $l$.
  
  \begin{proof}
    In order to show the lemma, it is shown that, for each state in a loop there must exist an automaton $\mathfrak{A}_i$ where both, the state is contained in and the state can not be merged with any other state.
    So let $q\in Q^l$ be the only state with $c(q)=i$ in a loop $l$. Since $q$ has the color $i$, the automaton $\mathfrak{A}_i$ accepts the words $w\in\Sigma^\star$ with $\delta^\star(q_0,w)=q$. In order to show that $|l_i^\prime|\leq|l|$ the following cases must be handled:
    \begin{enumertate}
      \item $q$ is the only node in $l$ that has color $i$
      \item $q$ is not the only node with color $i$
    \end{enumate}
    Assume the first case. In case $|l|=k^\prime$ and there is no other loop $l^\prime$ with $|l^\prime|<k$ the case is trivial. Since there is a shortest word $w\in\Sigma^\+$ such that $|w|=k$ and $\delta^+(q,w)=q$ with $q\in l$. Assume the loop $l_i^\prime$ in $\mathfrak{A}_i$, that is responsible for all words $w^\prime\in\Sigma^\star$ with $u\in\Sigma^\star$ and $v\in\Sigma^+,w=uv$ with $\delta^\star(q_0,u)=q$ and $\delta^+(q,v)=q$, has not as many states as $l$. So in there would be a shortest word $w_s\in\Sigma^+$ such that $\delta^+_i(q,w_s)=q$ but $\delta^+(q,w_s)\neq q$.
    Now let there be $q_1,\dots,q_j\in l^\prime_i$ with $c(q_1)=\dots=c(q_j)=i$. There meight be two cases that appear. In the first case, some nodes of the loop $l^\prime_i$ can be merge, in the other case no node can be merged.
    In case nodes can be merged it still must hold if $w\in \Sigma^\star$ and $\delta^\star(q,w)=q$ for $q\in l$ then also $\delta_i^\star(q,w)=q$. Thus if the states $q,q^\prime \in l$ are merged and $w^\prime\in\Sigma^+$ is the shortest word such that $\delta^+(q,w^\prime)=q^\prime$ holds, for $l$ the following must hold:
    \begin{displaymath}
    |l| \mod |w^\prime| = 0
    \end{displaymath}
    and every $|w|$-th state in $l$ must have the color $i$. 
  \end{proof}
\end{lemma}

\end{document}
