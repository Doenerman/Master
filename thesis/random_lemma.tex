\documentclass[10pt]{article}
\usepackage{geometry}                % See geometry.pdf to learn the layout options. There are lots.
\usepackage{blindtext}
\usepackage[parfill]{parskip}    % Activate to begin paragraphs with an empty line rather than an indent
\usepackage{tikz}
\usetikzlibrary{arrows,automata,shadows,positioning,shapes}
\usepackage{graphicx}
\usepackage{amssymb}
\usepackage{amsmath}
\usepackage{amsthm}
\usepackage{epstopdf}
\usepackage{hyperref}
\usepackage{listings}
\usepackage{subfiles}
\usepackage[utf8]{inputenc}
\usepackage{float}
\usepackage{tikz}
\usepackage{graphicx}
\usepackage{caption}
\usepackage{wrapfig}

\theoremstyle{definition}
\newtheorem{definition}{Definition}
\theoremstyle{plain}
\newtheorem{lemma}{Lemma}
\newtheorem{theorem}{Theorem}
\theoremstyle{remark}
\newtheorem{remark}{Remark}

\begin{document}

\begin{lemma}
  Let $\mathfrak{A}=(Q,\Sigma,q_0,\delta,c),c:Q\rightarrow \{0,\dots,k\}$ be a DPA and $(\mathfrak{A}_0,\dots,\mathfrak{A}_k)$ be its DFA representation. Let $Q^l\subseteq Q$ be the states of $\mathfrak{A}$ that are part of the loop and $Q_{i}^l$ be that states of $\mathfrak{A}_i$ that are part of a loop.
  If the automaton $\mathfrak{A}$ is minimal, then the following holds:
  \begin{displaymath}
    \bigcup\limits_{i\in\{0,\dots,k\} Q_i^{l}\geq Q^l
  \end{displaymath}
\end{lemma}

\end{document}
