\documentclass[10pt]{article}
\usepackage{geometry}                % See geometry.pdf to learn the layout options. There are lots.
\usepackage{blindtext}
\usepackage[parfill]{parskip}    % Activate to begin paragraphs with an empty line rather than an indent
\usepackage{tikz}
\usetikzlibrary{arrows,automata,shadows,positioning,shapes}
\usepackage{graphicx}
\usepackage{amssymb}
\usepackage{amsmath}
\usepackage{multicol}
\setlength{\columnsep}{1cm}
\usepackage{epstopdf}
\usepackage{hyperref}
\usepackage{listings}
\usepackage{subfiles}
\usepackage[utf8]{inputenc}
\usepackage{float}
\usepackage{tikz}
\usepackage{graphicx}
\usepackage{caption}
\usepackage{wrapfig}

\title{GAMS-Programming}
\begin{document}

\maketitle


  A casual gams extends of various section with a spacific order. In the
  following these sections are summed up with both, syntax and semantics.
  The \emph{GAMS}-program has one of the following styles.

  \begin{enumerate}
    \item \nameref{data}:
    \begin{enumerate}
      \item \nameref{data:declarations}
      \item \nameref{data:ParaDeclaNDef}
      \item \nameref{data:assingments}
      \item \nameref{data:displays}
    \end{enumerate}
    \item \nameref{model}
    \begin{enumerate}
      \item \nameref{model:VarDec}
      \item \nameref{model:EquationDec}
      \item \nameref{model:EquationDef}
      \item \nameref{model:ModelDefinition}
    \end{enumerate}
    \item \nameref{solution}
    \begin{enumerate}
      \item \nameref{solution:solve}
      \item \nameref{solution:displays}
    \end{enumerate}
  \end{enumerate}
  


    \textcolor{red}{The second style of a \emph{GAMS}-program is under
    construction}



  \section{Data}
  \label{data}
    \subsection{Set Declarations}
    \label{data:declarations}

    \subsection{Parameter declarations and Definitions}
    \label{data:ParaDeclaNDef}

    \subsection{Assignments}
    \label{data:assingments}

    \subsection{Displays}
    \label{data:displays}
    
  \section{Model}
  \label{model}
    \subsection{Variable Declaration}
    \label{model:VarDec}

    \subsection{Equation Declaration}
    \label{model:EquationDec}
    
    \subsection{Equation Definition}
    \label{model:EquationDef}

    \subsection{Model Definition}
    \label{model:ModelDefinition}

  \section{Solution}
  \label{solution}
    
    \subsection{Solve}
    \label{solution:solve}
    
    \subsection{Displays}
    \label{solution:displays}

\end{document}

