\documentclass[10pt]{article}
\usepackage{geometry}                % See geometry.pdf to learn the layout options. There are lots.
\usepackage{blindtext}
\usepackage[parfill]{parskip}    % Activate to begin paragraphs with an empty line rather than an indent
\usepackage{tikz}
\usetikzlibrary{arrows,automata,shadows,positioning,shapes}
\usepackage{graphicx}
\usepackage{amssymb}
\usepackage{amsmath}
\usepackage{multicol}
\setlength{\columnsep}{1cm}
\usepackage{epstopdf}
\usepackage{hyperref}
\usepackage{listings}
\usepackage{ascii}
\usepackage{subfiles}
\usepackage[utf8]{inputenc}
\usepackage{float}
\usepackage{tikz}
\usepackage{graphicx}
\usepackage{caption}
\usepackage{wrapfig}

\title{\text{\asciifamily GAMS-Programming}}
\begin{document}

\maketitle


  A casual gams extends of various section with a spacific order. In the
  following these sections are summed up with both, syntax and semantics.
  The \text{\asciifamily GAMS}-program has one of the following styles.

  \begin{enumerate}
    \item \nameref{data}:
    \begin{enumerate}
      \item \nameref{data:declarations}
      \item \nameref{data:ParaDeclaNDef}
      \item \nameref{data:assingments}
      \item \nameref{data:displays}
    \end{enumerate}
    \item \nameref{model}
    \begin{enumerate}
      \item \nameref{model:VarDec}
      \item \nameref{model:EquationDec}
      \item \nameref{model:EquationDef}
      \item \nameref{model:ModelDefinition}
    \end{enumerate}
    \item \nameref{solution}
    \begin{enumerate}
      \item \nameref{solution:solve}
      \item \nameref{solution:displays}
    \end{enumerate}
  \end{enumerate}
  


  \textcolor{red}{The second style of a \text{\asciifamily GAMS}-program is under
    construction}



  \section{Data}
  \label{data}
    \subsection{Set Declarations}
    \label{data:declarations}
      In the \emph{set}-declaration the different sets that are needed to solve
      the LP are declared. This means here you name all the different sets that
      are requiered to solve the LP. 
      \begin{lstlisting}
        Sets
            i   1 / 2 / 3 /
            j   berlin / aachen / hamburg / ;
      \end{lstlisting}

      In \text{\asciifamily GAMS} there are some rules for \text{\asciifamily Sets}.
      \begin{enumerate}
        \item after each element of a set a / must follow
        \item end the \text{\asciifamily {Sets}}-section with a ";"
      \end{enumerate}

      

    \subsection{Parameter declarations and Definitions}
    \label{data:ParaDeclaNDef}
      In the \emph{parameter}-declaration functions of the LP are declared and
      defined. These functions mostly depend on at least one set. For the input
      (set elements) the \emph{parameter} are fixed values.

      \begin{lstlisting}
        Parameters
            f(x)    some function
              /     x1     1
                    x2     2 / ;
      \end{lstlisting}
      The rules for \text{\asciifamily Parameters} are the following:
      \begin{enumerate}
        \item split different paramteters with /
        \item end the \text{\asciifamily Parameters}-section with ";"
      \end{enumerate}

    \subsection{Assignments}
    \label{data:assingments}
      \textcolor{red}{Under Construction}

    \subsection{Displays}
    \label{data:displays}
      \textcolor{red}{Under Construction}
    
  \section{Model}
  \label{model}
    \subsection{Variable Declaration}
    \label{model:VarDec}
    In this part of the program \emph{variables} can be declared. The
    \emph{variables} represent the \emph{decision variables} of the LP.
      \begin{lstlisting}
        Variables
            x(i)  some function depending on i
      \end{lstlisting}

    \subsection{Equation Declaration}
    \label{model:EquationDec}
      The \emph{equations} of a \text{\asciifamily GAMS}-program represent the
      constraints of the LP. But before defining the different constraints they
      are named. In case all equations are named a ";" marks the point where the
      equations are defined. 
      \begin{lstlisting}
        Equations
            equationName     A short description about the equation;
      \end{lstlisting}
    
    \subsection{Equation Definition}
    \label{model:EquationDef}
      After a \emph{Equation} is declared it can be defined. To do so the
      following three expressions can be used:
      \begin{table}[h]
        \centering
        \begin{tabular}{ c | c }
          $\leq$ & =l= \\
          $=$ & =e= \\
          $\geq$ & =g= \\
        \end{tabular}
      \end{table}
      The definition of the equation are directly below the declaration of the
      \emph{equation}. The ";" is used to seperated the different equation. In
      order to define a equation the following syntax is used:
      \begin{lstlisting}
            equationName ..   z =g= 0;
      \end{lstlisting}


    \subsection{Model Definition}
    \label{model:ModelDefinition}

  \section{Solution}
  \label{solution}
    
    \subsection{Solve}
    \label{solution:solve}
    
    \subsection{Displays}
    \label{solution:displays}

\end{document}

