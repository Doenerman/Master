\documentclass[10pt]{article}
\usepackage{geometry}                % See geometry.pdf to learn the layout options. There are lots.
\usepackage{blindtext}
\usepackage[parfill]{parskip}    % Activate to begin paragraphs with an empty line rather than an indent
\usepackage{tikz}
\usetikzlibrary{arrows,automata,shadows,positioning,shapes}
\usepackage{graphicx}
\usepackage{amssymb}
\usepackage{amsmath}
\usepackage{epstopdf}
\usepackage{hyperref}
\usepackage{listings}
\usepackage{subfiles}
\usepackage[utf8]{inputenc}
\usepackage{float}
\usepackage{tikz}
\usepackage{graphicx}
\usepackage{caption}
\usepackage{wrapfig}

\begin{document}

  \section*{Aufgabe 19}
    \subsection*{a}
      \begin{displaymath}
        \begin{array}{ r r c l l }
          \min & d_{max} & & & \\
          s.t. & \sum\limits_{(i,j)\in \delta^{+}(i)} x_{ij} & = & 1 & \forall 
            i\in \{1,..,n\}\\
               & \sum\limits_{(i,j)\in \delta^{-}(i)} x_{ji} & = & 1 & \forall 
            i\in \{1,..,n\}\\
            & \sum\limits_{(i,j)\in \delta^{+}(S)} x_{ij} & \geq &  1 & \forall
            S\subsetneq V, S\neq \emptyset \\
               & x_{ji}\cdot d_{ji} &\leq& d_{max} & \forall i,j \in
            \{1,...,n\}, i\neq j \\
        \end{array}
      \end{displaymath}
      \begin{displaymath}
        x_{ij}\in\{0,1\}, \delta^{+}(S)=\{(i,j)\in A:i\in S, j\not\in S\},
        \delta^{-}(S)=\{(i,j)\in A:j\in S, i\not\in S\}
      \end{displaymath}

      Die boolsche Variable $x_{ij}$ ist genau dann $1$, wenn ein Bus von der
      Station $i\in \{1,...n\}$ nach $j \in \{1,...,n\}$ fährt, wobei $i\neq j$
      gilt.


      Die ersten drei Nebenbedingungen stellen sicher, dass es sich um genau
      einen Kreis in dem Graph gibt und dass sich keine Teilgraphen bilden
      lassen zwischen denen es keine Verbindung gibt, so dass jeder Knoten genau
      einmal angefahren wird.
      Die letzte Nebenbedingung sucht die längste Kanten entlang des Kreises,
      welche in der Zielfunktion minimiert wird.



    \subsection*{b}
      \begin{displaymath}
        \begin{array}{ r r c l l }
          \min & d_{max} & & & \\
          s.t. & \sum\limits_{(i,j)\in \delta^{+}(i)} x_{ij} & = & 1 & \forall 
            i\in \{1,..,n\}\\
               & \sum\limits_{(j,i)\in \delta^{-}(i)} x_{ji} & = & 1 & \forall 
            i\in \{1,..,n\}\\
              & \sum\limits_{(i,j)\in \delta^{+}(S)} x_{ij} & \geq &  1 & \forall
            S\subsetneq V, S\neq \emptyset \\
            & x_{ij} + x_{jk} & \leq &1 + x_{ik}^{\prime} & \forall
                i,j,k \in \{1,...,n\} \\
            & x_{ij}^{\prime}\cdot d_{ij}^{\prime} &\leq&
              d_{max} & \forall i,j \in \{1,...,n\}, i\neq j \\
            & d_{ij} + d_{jk} &=& d^{\prime}_{ik} & \forall i,j,k\in V, i\neq j
                \neq k
        \end{array}
      \end{displaymath}
      \begin{displaymath}
        x_{ij},x_{ij}^{\prime}\in\{0,1\}, d_{ij}^{\prime} \geq 0
      \end{displaymath}
      Sei das TSP ein Graph $G=(V,A)$
      Die boolsche Variable $x_{ij}^{\prime} \in \{0,1\}$ ist genau dann $1$,
      wenn ein Bus von der Station $i \in \{1,...,n\}$ über eine weitere Station 
      zu der Station $j\in \{1,...,n\}$ fährt. Die Variable $d_{ij}^{\prime} \in
      \mathbb{R}^{+}$ stellt die Distanz einer Strecke von $i \in \{1,...,n\}$
      nach $k\in\{1,...,n\}$ mit genau einer beliebigen Station $j\in
      \{1,...,n\}$ dazwischen. 

      Die vierte
      Nebenbedingung sucht nach Stationen, die aufeinander folgen. Hierfür wird
      die neue Variable $x_{ij}^{\prime}$ wie oben beschrieben verwendet.
      Sollte von $i$ über $j$ nach $k$ gefahren werden, mit
      $i,j,k\in\{1,...,n\}$, dann gilt $x_{ij}=x_{jk}=1$, also $x_{ij}+x_{jk}=2$
      woraus $x_{ik}^{\prime}=1$ folgt. Sollte $x_{ij} \neq x_{jk}$ oder
      $x_{ij}=x_{jk}=0$ sein, so bilden die Stationen offensichtlich keine
      Verbindung und $x_{ik}^{\prime}$ muss nicht auf $1$ gesetzt werden. Da es
      sich hierbei um ein Minimierungsproblem handelt und $x_{ik}^{\prime}$
      indirekt mit in die Zielfunktion einfließt wird dieses dann auch nicht auf
      $1$ gesetzt.
      Die letzte Nebenbedingung berechnet alle Strecken bei denen eine
      Bushaltestelle übersprungen wird.

      Die Idee des LPs besteht also darin zu jeder Station eine Vorgängerstation
      zu finden. Da es sich um eine Rundtour handelt gibt es keine Ausnahme und
      es ist ebenso egal, bei welcher Station begonnen werden muss.

      
  \section*{Aufgabe 20}
    \subsection*{a}
      \begin{displaymath}
        \begin{array}{ l l c r r }
          \min & \sum\limits_{(i,j)\in A} x_{ij}\cdot d_{ij} & & & \\
          s.t. & \sum\limits_{(i,j)\in \delta(i)} x_{ij}
          +\sum\limits_{(j,i)\in\delta(i)} x_{ji}& \geq & 2\cdot z_{i} & 
          \forall i \in \{1,...,n\}\\
          & \sum\limits_{(i,j)\in \delta^{+}(S)} x_{ij} & \geq &  2 (z_{h} +
          z_{l} -1  & \forall S\subsetneq V, S\neq \emptyset, \forall h\in S,
          \forall l\not\in S \\
            & \sum\limits_{(1,i)\in\delta(i)} x_{1i} +
            \sum\limits_{(i,1)\in\delta(i)} x_{i1} &=& 2 & \forall i\in \\
            & x_{ij} & \leq & z_{j} & \forall j,i\in V \\
            &\sum\limits_{i\in L_{k}} z_{i} & \geq & d_{k} & \forall k\in\{1,...,m\}
        \end{array}
      \end{displaymath}
      \begin{displaymath}
        z_{i}\in\{0,1\},x_{ij}\in\{0,1\}, \delta^{+}(S)=\{(i,j)\in A:i\in S, j\not\in S\}
      \end{displaymath}

      Sei das TSP auf dem Graphen $G=(V,A)$.

      Die boolsche Variable $x_{ij}$ ist genau dann $1$, wenn in der Stadt $i$
      ein Konzert gespielt wird und als nächstes in der Stadt $j$, wodurch die
      Kosten $d_{ij}$ anfallen.
      Die boolsche Variable $z_{i}$ ist genau dann $1$, wenn während der Tour
      in der Stadt $i$ ein Konzert gespielt wird.

      In der Zielfunktion werden alle Kosten aufsummiert die tatsächlich
      anfallen, also werden Kosten genau dann aufsummiert, wenn die Strecke
      $d_{ij}$ tatsächlich genutzt wird.
      Die letzte Nebenbedingung sagt aus, dass die Anzahl der Konzert in einem
      Land $k$ mindestens $d_{k}$ sein muss. Hierfür wird über alle Städte in
      $k$ summiert und für jedes in Land $k$ statt finde Konzert $1$ addiert.
      Die erste Nebenbedingung legt fest, dass zu jeder Stadt, in der ein
      Konzert gespielt wird, zwei Kanten vorliegen, damit man zu dem Konzert
      gelangen kann und die Stadt im Anschluss wieder verlassen kann. Die zweite
      Nebenbedingung ist wie in der obrigen Aufgabe zu verstehen, dass es nur
      einen Kreis gibt und keinen kleineren unabhängigen Kreise. Die dritte
      Nebenbedingung legt fest, dass die Stadt $1$ definitiv besucht wird. Wann
      diese in dem Kreis besucht wird ist egal, die Tour kann mitten im Kreis
      starten. Der Rest der Tour kann dann aus den $x_{ij}$ hergeleitet werden.

    \subsection*{b}
      \begin{displaymath}
        \begin{array}{ l l c r r }
          \min & \sum\limits_{(i,j)\in A} x_{ij}\cdot d_{ij} & & & \\
          s.t. & \sum\limits_{(i,j)\in \delta^{+}(i)} x_{ij} &=& z_{i} & \forall
                i\in V \\
          &\sum\limits_{(j,i)\in\delta^{-}(i)} x_{ij}& = & z_{i} & \forall 
            i\in V\\
            & \sum\limits_{(i,j)\in \delta^{+}(S)} x_{ji} & \geq &  1 & \forall
            S\subsetneq V, S\neq \emptyset \\
            & \sum\limits_{(1,i)\in\delta^{+}(i)} x_{1i} & = & 1 & \\
            &\sum\limits_{(1,i)\in\delta^{-}(i)} x_{i1} &=& 1 & \\
            & x_{ij} & \leq & z_{j} & \forall j,i\in V \\
            &\sum\limits_{i\in L_{k}} z_{i} & \geq & d_{k} & \forall k\in\{1,...,m\}
        \end{array}
      \end{displaymath}
      \begin{displaymath}
        z_{i}\in\{0,1\},x_{ij}\in\{0,1\}, \delta^{+}(S)=\{(i,j)\in A:i\in S, j\not\in S\}
      \end{displaymath}

      Die ersten Nebenbedingungen sind wie im im obigen Aufgabenteil, wobei die
      dritte Bedingung sagt, dass Stadt $1$ definitiv mit in der Tour sein muss.
      Es muss nicht gesagt werden, dass diese explizit als erstes angefahren
      werden muss. Da diese auf dem Kreis liegt welcher der Tour entspricht
      genügt es bei der Planung die Tour dort zu beginnen.
      Durch den gerichteten Graphen müssen die Kanten, die an einem Knoten
      anliegen näher beschrieben werden, ob es sich dabei um eine eingehende
      koder eine ausgehnde Kante handelt. Dies betrifft Nebenbedingung
      1,2,4 und 5. Die Grundidee folgt aus dem LP aus dem vorherigen
      Aufgabenteil.


\end{document}

