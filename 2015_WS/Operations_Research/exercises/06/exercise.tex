\documentclass[10pt]{article}
\usepackage{geometry}                % See geometry.pdf to learn the layout options. There are lots.
\usepackage{blindtext}
\usepackage[parfill]{parskip}    % Activate to begin paragraphs with an empty line rather than an indent
\usepackage{tikz}
\usetikzlibrary{arrows,automata,shadows,positioning,shapes}
\usepackage{graphicx}
\usepackage{amssymb}
\usepackage{amsmath}
\usepackage{epstopdf}
\usepackage{hyperref}
\usepackage{listings}
\usepackage{subfiles}
\usepackage[utf8]{inputenc}
\usepackage{float}
\usepackage{tikz}
\usepackage{graphicx}
\usepackage{caption}
\usepackage{wrapfig}

\begin{document}

  \section*{Aufgabe 19}
    \subsection*{a}
      \begin{displaymath}
        \begin{array}{ r r c l l }
          \min & d_{max} & & & \\
          s.t. & \sum\limits_{j\in \{1,...,n\}} x_{ij} & = & 1 & \forall 
            i\in \{1,..,n\}\\
            & x_{ji}\cdot d_{ji} &\leq& d_{max} & \forall i,j \in \{1,...,n\}, i\neq j \\
        \end{array}
      \end{displaymath}
      \begin{displaymath}
        x_{ij}\in\{0,1\}
      \end{displaymath}

      Die boolsche Variable $x_{ij}$ ist genau dann $1$, wenn ein Bus von der
      Station $i\in \{1,...n\}$ nach $j \in \{1,...,n\}$ fährt, wobei $i\neq j$
      gilt.

      Die erste Nebenbedingung garantiert, dass jede Station genau einmal angefahren
      wird. In der zweiten Nebenbedingung wird die längste Distanz unter den
      verwendeten Distanzen gesucht. Welche dann, laut Zielfunktion, minimiert
      wird. Da diese Nebenbedingung unabhängig über für alle Kombinationen aus
      $i,j\in \{1,...,n\}$ geprüft wird, ist es nicht notwendig eine weitere
      Nebenbedingung für $d_{ji}$ hinzuzufügen.


    \subsection*{b}
      \begin{displaymath}
        \begin{array}{ r r c l l }
          \min & d_{max} & & & \\
          s.t. & \sum\limits_{j\in \{1,...,n\}} x_{ij} & = & 1 & \forall 
            i\in \{1,..,n\}\\
            & x_{ij}\cdot + x_{jk} & \leq &1 + x_{ik}^{\prime} & \forall
                i,j,k \in \{1,...,n\} \\
            & x_{ij}^{\prime}\cdot d_{ij}^{\prime} + x_{jk} \cdot d_{jk}&\leq&
              d_{max} & \forall i,j,k \in \{1,...,n\}, i\neq j \\
        \end{array}
      \end{displaymath}
      \begin{displaymath}
        x_{ij},x_{ij}^{\prime}\in\{0,1\}, d_{ij}^{\prime} \geq 0
      \end{displaymath}

      Die boolsche Variable $x_{ij}^{\prime} \in \{0,1\}$ ist genau dann $1$,
      wenn ein Bus von der Station $i \in \{1,...,n\}$ über eine weitere Station 
      zu der Station $j\in \{1,...,n\}$ fährt. Die Variable $d_{ij}^{\prime} \in
      \mathbb{R}^{+}$ stellt die Distanz einer Strecke von $i \in \{1,...,n\}$
      nach $k\in\{1,...,n\}$ mit genau einer beliebigen Station $j\in
      \{1,...,n\}$ dazwischen. 

      Die ersten Nebenbedingung ist wie im im obrigen Aufgabenteil. Die zweite
      Nebenbedingung sucht nacht Stationen, die aufeinander folgen. Hierfür wird
      die neue Variable $x_{ij}^{\prime}$ wie oben beschrieben verwendet.
      Sollten sollte von $i$ über $j$ nach $k$ gefahren werden, mit
      $i,j,k\in\{1,...,n\}$, dann gilt $x_{ij}=x_{jk}=1$, also $x_{ij}+x_{jk}=2$
      woraus $x_{ik}^{\prime}=1$ folgt. Sollte $x_{ij} \neq x_{jk}$ oder
      $x_{ij}=x_{jk}=0$ sein, so bilden die Stationen offensichtlich keine
      Verbindung und $x_{ik}^{\prime}$ muss nicht auf $1$ gesetzt werden. Da es
      sich hierbei um ein Minimierungsproblem handelt und $x_{ik}^{\prime}$
      indirekt mit in die Zielfunktion einfließt wird dieses dann auch nicht auf
      $1$ gesetzt.

      Die Idee des LPs besteht also darin zu jeder Station eine Vorgängerstation
      zu finden. Da es sich um eine Rundtour handelt gibt es keine Ausnahme und
      es ist ebenso egal, bei welcher Station begonnen werden muss.

      
  \section*{Aufgabe 20}
    \subsection*{a}
      \begin{displaymath}
        \begin{array}{ l l c r r }
          \min & \sum\limits_{i\in\{1,...,n\}}\sum\limits_{j\in
                \{1,...,m\},j\neq i} x_{ij}\cdot d_{ij} & & & \\
          s.t. & \sum\limits_{j\in L_{k}} x_{j}& \geq & d_{k} & \forall k \in
                  \{1,...,m\}\\
               & \sum\limits_{i\in\{1,...,n\}} x_{1i} &=& 1 & \\
               & \sum\limits_{i\in\{1,...,n\}} x_{i1} &=& 1 & \\
               & \sum\limits_{i\in\{1,...,n\}} x_{ij} &=& x_{j} & \forall
                  j\in\{1,...,n\} \\
               & \sum\limits_{i\in\{1,...,n\}} x_{ji} &=& x_{j} & \forall
                  j\in\{1,...,n\} \\
        \end{array}
      \end{displaymath}

      Die boolsche Variable $x_{ij}$ ist genau dann $1$, wenn in der Stadt $i$
      ein Konzert gespielt wird und als nächstes in der Stadt $j$, wodurch die
      Kosten $d_{ij}$ anfallen.
      Die boolsche Variable $x_{i}$ ist genau dann $1$, wenn während der Tour
      in der Stadt $i$ ein Konzert gespielt wird.

      In der Zielfunktion werden alle Kosten aufsummiert die tatsächlich
      anfallen, also werden Kosten genau dann aufsummiert, wenn die Strecke
      $d_{ij}$ tatsächlich genutzt wird.
      Die erste Nebenbedingung sagt aus, dass die Anzahl der Konzert in einem
      Land $k$ mindestens $d_{k}$ sein muss. Hierfür wird über alle Städte in
      $k$ summiert und für jedes in Land $k$ stattfinde Konzert $1$ addiert.
      Die zweite Nebenbedingung legt die Stadt $1$ als Startpunkt der Tour fest.
      Die dritte Nebenbedingung legt die Stadt $1$ als die Stadt fest, in
      welcher die Tour endet.

    \subsection*{b}
      In dem LP aus dem obrigen Aufgabenteil muss nur die Funktion $d_{ij}$
      durch $d_{ij}^{\prime}$ ersetzt werden. Da im obrigen Aufgabenteil nicht
      von $d_{ij}=d_{ji}$ ausgegangen worden ist, muss an dieser Stelle somit
      keine Veränderung vorgenommen werden.


  \section*{Aufgabe 21}
    Zur einfacherern Lösung des Problems und Erstellung des \text{
    GAMS}-Programms wird das LP zunächst formal mathematisch aufgeschrieben.

    Sets:
    \begin{enumerate}
        \item $t$ Zeitpunkte
        \item $j$ Jobs
    \end{enumerate}

    \begin{displaymath}
      \begin{array}{ r r c l l }
        \min & f_{max} & & & \\
        s.t. & s_{j} & \geq & r_{j} & \forall j\in J \\
             & s_{j} & \geq & s_{prec(j)} + p_{prec(j)} & \forall j \in J \\
             & f_{j} & = & s_{j} + p_{j} & \forall j\in J\\
             & x_{j}(t) & = & 1 & \forall j\in J, s_{j} \leq t \leq f_{j}\\
             & f_{j} & \leq & f_{max} & \forall j \in J \\
             & \sum\limits_{j\in J} x_{j}(t) &=& 1 & \forall t\in T \\
      \end{array}
    \end{displaymath}
    \begin{displaymath}
      s_{j}\geq 0, f_{j}\geq 0,x_{j}(t)\in\{0,1\}
    \end{displaymath}
\end{document}

