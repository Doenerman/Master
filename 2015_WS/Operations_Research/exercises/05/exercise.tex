\documentclass[10pt]{article}
\usepackage{geometry}                % See geometry.pdf to learn the layout options. There are lots.
\usepackage{blindtext}
\usepackage[parfill]{parskip}    % Activate to begin paragraphs with an empty line rather than an indent
\usepackage{tikz}
\usetikzlibrary{arrows,automata,shadows,positioning,shapes}
\usepackage{graphicx}
\usepackage{amssymb}
\usepackage{amsmath}
\usepackage{epstopdf}
\usepackage{hyperref}
\usepackage{listings}
\usepackage{subfiles}
\usepackage[utf8]{inputenc}
\usepackage{float}
\usepackage{tikz}
\usepackage{graphicx}
\usepackage{caption}
\usepackage{wrapfig}

\begin{document}

\section*{Aufgabe 1}
  \subsection*{a}
    \begin{displaymath}
      \begin{array}{ r r c l l}
        \min & e & & & \\
        s.t. & \sum\limits_{j\in J} x_{ij} & \leq & b_i & \forall i\in I \\
             & \sum\limits_{i\in I} x_{ij} & \geq & d_i & \forall j\in J \\
             & x_{ij} & \leq & e & \forall i\in I\, \forall j\in J
      \end{array}
    \end{displaymath}
    \begin{enumerate}
      \item $i\in I$ Zentralbanken
      \item $j\in J$ Geschäftsbanken
      \item $x_{ij}\in \mathbb{N}$ Geld von Zentralbank $i\in I$ 
            an Geschäftsbank $j\in J$
      \item $b_i\in \mathbb{N}$ maximale druckbares Geld in Zentralbank $i\in I$
      \item $d_i\in \mathbb{N}$ minmale Menge an Geld, die Geschäftsbank $j\in J$
            erhalten soll
    \end{enumerate}

  \subsection*{b}
    \begin{displaymath}
      \begin{array}{ r r c l l }
        \min & \sum\limits_{i\in I}\sum\limits_{j\in J} b_{ij} & & & \\
        s.t. & \sum\limits_{j\in J}b_{ij}\cdot x_{ij} & \leq & b_i & \forall i\in I \\
             & \sum\limits_{i\in I}b_{ij}\cdot x_{ij} & \geq & d_i & \forall j\in J \\
      \end{array}
    \end{displaymath}

    Zusätzliche Variable zu Aufgabe a
    \begin{enumerate}
      \item $b_{ij} \in \{0,1\}$ ist genau dann $1$, wenn Zentralbank $i\in I$
        an Geschäftsbank $j\in J$ liefert
    \end{enumerate}



\end{document}

