\documentclass[10pt]{article}
\usepackage{geometry}                % See geometry.pdf to learn the layout options. There are lots.
\usepackage{blindtext}
\usepackage[parfill]{parskip}    % Activate to begin paragraphs with an empty line rather than an indent
\usepackage{tikz}
\usetikzlibrary{arrows,automata,shadows,positioning,shapes}
\usepackage{graphicx}
\usepackage{amssymb}
\usepackage{amsmath}
\usepackage{epstopdf}
\usepackage{hyperref}
\usepackage{listings}
\usepackage{subfiles}
\usepackage[utf8]{inputenc}
\usepackage{float}
\usepackage{tikz}
\usepackage{graphicx}
\usepackage{caption}
\usepackage{wrapfig}

\begin{document}

\section*{Aufgabe 8}
  \begin{equation}
    \begin{array}{r l l}
      \max & \sum\limits_{i=1}^{8} \sum\limits_{j=1}^{8} x_{i,j}& \\
      s.t.& x_{i,j}(x_{i-2,j-1} + x_{i-1,j-2} + x_{i+1,j-1} + x_{i+2,j-1} & \\
          & x_{i-2,j+1} + x_{i-1,j+2} + x_{i+1,j+1} + x_{i+2,j+1})=0 & \forall i,
      \forall j \\
          & x_{i,j} \geq 0&
    \end{array}
  \end{equation}

  Die boolche Variable $x_{i,j}$ besitz den Wert $1$ genau dann,
  wenn an der Position $(i,j)$ auf dem Schachbrett ein Springer steht. Für alle
  $i,j\leq 0,\,i,j\geq 8$ folgt $x_{i,j}=0$. Dies dient für eine vereinfachte
  Darstellung des LPs.

  Die erste Gleichung legt fest, dass wenn an der Stelle $(i,j)$ ein Springer steht
  an den umliegenden Springerpositionen kein Springer steht, oder dass an der
  Position $(i,j)$ sonst kein Springer steht. Ein Teil des Produktes muss $0$
  sein um diese Gleichung für jedes Feld zu erfüllen.

  In der Zielfunktion sollen möglichst viele Springer auf das Feld gestellt
  werden, weshalb über alle Positionen auf dem Feld die $x_{i,j}$ summiert
  werden.


\section*{Aufgabe 9}
  \begin{equation}
    \begin{array}{r l l}
      \min \sum_{i=0}^{n} y_{i}
      s.t. & x_{v,j} \cdot x_{u,j} = 0 & \forall (u,v)\in E,\, \forall j\in
            \{1,...,n\} \\
            & x_{v,j} \leq y_{j} & \forall v\in V, \forall j\in \{0,...,n\}
    \end{array}
  \end{equation}

  Die boolsche Variable $x_{v,j}$ ist genau dann $1$, wenn der Knoten $v$ die
  Farbe $j$ besitzt.
  Die boolsche Variable $y_{j}$ ist genau dann $1$, wenn es einen Knoten gibt,
  der die Farbe $j$ besitzt.

  Die erste Gleichung sorgt dafür, dass es keine zwei Knoten gibt, die die
  gleiche Farbe besitzen.

  Die zweite Gleichung garantiert, dass falls ein Knoten die Farbe $j$ besitzt
  die Variable $y$ auch mindestens den Wert $1$ besitzt.


\section*{Aufgabe 10}
  \begin{equation}
    \begin{array}{r l l}
      \min  & 1.8\cdot y_{s1} + 2.5\cdot y_{s2} + 2.5\cdot y_{s3} &\\
            & + 0.6\cdot x_{s1,o1} + 0.5\cdot x_{s1,o2} +3.9\cdot x_{s1,o3} 
                + 3.6\cdot x_{s1,o4} & \\
            & + 1.4\cdot x_{s2,o1} + 2\cdot x_{s2,o2} + 0.9\cdot x_{s2,o3}
                + 2.8\cdot x_{s2,o4} & \\
            & + 2.4\cdot x_{s3,o1} + 1.8\cdot x_{s3,o2} + 1.2 \cdot x_{s3,o3} 
              + 1.1 \cdot x_{s3,o4} & \\
      s.t. & \sum\limits_{i\in \{s1,s2,s3\}} x_{i,o} \geq 1 & \forall o \in 
              \{o1, o2, o3\} \\
           & \sum\limits_{o\in \{o1,o2.o3\}} x_{i,o} \leq 2 & \forall i\in
              \{s1,s2,s3\} \\
           & x_{i,o} \leq y_{i} & \forall i \in \{s1,s2,s3\}, \forall o\in 
              \{o1,o2,o3\}

    \end{array}
  \end{equation}

  Die boolsche Variable $x_{i,o}$, wobei $i$ die Raffinerie bezeichnet und $o$
  das Ölfeld, ist genau dann $1$, wenn die Raffinerie $i$ gebaut ist, um das
  Ölfeld $o$ zu nutzen.

  Die boolsche Variable $y_{i}$ ist genau dann $1$, wenn die Raffinerie $i \in
  \{s1,s2,s3\}$ gebaut wird.
  
  Die erste Gleichung fordert, dass jedes Ölfeld mit wenigstens einer Raffinerie
  verbunden ist.
  Die zweite Gleichung begrenzt die Anzahl der Ölfelder die mit einer Raffinerie
  verbunden mit zwei begrenzt ist.

\end{document}

