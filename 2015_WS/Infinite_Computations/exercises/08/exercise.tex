\documentclass[10pt]{article}
\usepackage{geometry}                % See geometry.pdf to learn the layout options. There are lots.
\usepackage{blindtext}
\usepackage[parfill]{parskip}    % Activate to begin paragraphs with an empty line rather than an indent
\usepackage{tikz}
\usetikzlibrary{arrows,automata,shadows,positioning,shapes}
\usepackage{graphicx}
\usepackage{amssymb}
\usepackage{amsmath}
\usepackage{amsthm}
\usepackage{epstopdf}
\usepackage{hyperref}
\usepackage{listings}
\usepackage{subfiles}
\usepackage[utf8]{inputenc}
\usepackage{float}
\usepackage{tikz}
\usepackage{graphicx}
\usepackage{caption}
\usepackage{wrapfig}

\begin{document}

  \section*{Exercise 36}
    \begin{displaymath}
      \varphi^{\prime}(X_{1})=0\subseteq X_{1} \land \forall X \exists Y (Sing(X) \land Sing(Y) \land
      X\subseteq X_{1} \land Succ(X,Y) \rightarrow \neg ( Y \subseteq X_{1}))
    \end{displaymath}

  \section*{Exercise 37}
    \subsection*{a}
      \begin{displaymath}
        \forall s \exists x(
          X_{3}(x) \land
          s < x \rightarrow
            (
              (X_{0}(s) \rightarrow X_{1}(s+1)) 
              \lor (X_{1}(s) \rightarrow X_{2}(s+1)) 
              \lor (s+1<x \land X_{2}(s) \rightarrow X_{0}(s+1))
            )
        )
        \land \forall x \exists y (x+1=y)
      \end{displaymath}
    \subsection*{b}
      \begin{displaymath}
        \left(
          \begin{array}{c}
            * \\
            0
          \end{array}
        \right)^{*} \cdot
        \left(
          \begin{array}{c}
            0 \\
            0
          \end{array}
        \right) \cdot
        \left(
          \begin{array}{c}
            * \\
            0
          \end{array}
        \right)^{*} \cdot
        \left(
          \begin{array}{c}
            * \\
            1
          \end{array}
        \right)^{*}
        \left(
          \begin{array}{c}
            1 \\
            1
          \end{array}
        \right) \cdot
        \left(
          \begin{array}{c}
            * \\
            1
          \end{array}
        \right)^{*}
      \end{displaymath}
      The first row of the formula says that if there is an element in $X_{2}$
      all successors are in $X_{2}$ as well. The second row of the formula name
      the two elements that must occur ($
        \left(
          \begin{array}{c}
            1 \\
            1
          \end{array}
        \right),
        \left(
          \begin{array}{c}
            0 \\
            0
          \end{array}
        \right)$).

  \section*{Exercise 38}
    \subsection*{a}
      \begin{displaymath}
        \begin{array}{c c l}
          \varphi(X_{1}) & = & 
            \exists Y_{1} Y_{2} Y_{3} (
            Partition(Y_{1},Y_{2},Y_{3}) \land Y_{1}(0)
            \forall t ( \\
          &  &  (Y_{1}(t) \land X_{1}(t) \land Y_{2}(t^\prime)) \lor\\ 
          &  &  (Y_{2}(t) \land \neg X_{1}(t) \land Y_{2}(t^\prime)) \lor\\ 
          &  &  (Y_{2}(t) \land X_{1}(t) \land Y_{3}(t^\prime)) \lor     \\ 
          &  &  (Y_{3}(t) \land \neg X_{1}(t) \land Y_{2}(t^\prime))     \\ 
          &  &  \forall s \exists t ( s < t \land Y_{3}(t) ))\\
        \end{array}
      \end{displaymath}


\end{document}

