\documentclass[10pt]{article}
\usepackage{geometry}                % See geometry.pdf to learn the layout options. There are lots.
\usepackage{blindtext}
\usepackage[parfill]{parskip}    % Activate to begin paragraphs with an empty line rather than an indent
\usepackage{tikz}
\usetikzlibrary{arrows,automata,shadows,positioning,shapes}
\usepackage{graphicx}
\usepackage{amssymb}
\usepackage{amsmath}
\usepackage{amsthm}
\usepackage{epstopdf}
\usepackage{hyperref}
\usepackage{listings}
\usepackage{subfiles}
\usepackage[utf8]{inputenc}
\usepackage{float}
\usepackage{tikz}
\usepackage{graphicx}
\usepackage{caption}
\usepackage{wrapfig}

\begin{document}

\section*{Exercise 1}
  \subsection*{a)}
  \subsection*{b)}
    \emph{Scan Conversion} is the process where the different polygons converted
    to pixels.
  \subsection*{c)}
    Only a quarter of the circle needs to be specified. The quarter can be
    mirrored twice in order to calculated the complete circle. 
    
    The implicit function $F(x,y)$ for the given segment of the circle is the
    following
    \begin{displaymath}
      F(x,y)= x^2 + y^2 - r
    \end{displaymath}
    where $r$ is about $2.8$ because the origin of the circle is in the middle
    and a complete quarter of the circle is given. The value of $r$ is extracted
    of the picture.
  
    \textbf{1. Step} \emph{Initialize d}
    \begin{displaymath}
      \begin{array}{ r c l }
        d & = &F(M) \\
          & = &F(x+1,y-\frac{1}{2}) \\
          & = &(x+1)^2+(y-\frac{1}{2})^2-r
      \end{array}
    \end{displaymath}
    In the first step is $x_1=0$ and $y_1=3$. With $r=2.8$, $d=1^2+(1.5)^2-2.8$, so
    $d=0.45>0$. The algorithm chooses (E).

    \textbf{2. Step}
    \begin{displaymath}
      \begin{array}{ l c r }
        d_{new} & = & F(x_1+2,y_1-\frac{1}{2}) \\
                & = & 2^2+1.5^2-2.8 \\
                & = & 3.45
      \end{array}
    \end{displaymath}



\end{document}

