\documentclass[10pt]{article}
\usepackage{geometry}                % See geometry.pdf to learn the layout options. There are lots.
\usepackage{blindtext}
\usepackage[parfill]{parskip}    % Activate to begin paragraphs with an empty line rather than an indent
\usepackage{tikz}
\usetikzlibrary{arrows,automata,shadows,positioning,shapes}
\usepackage{graphicx}
\usepackage{amssymb}
\usepackage{amsmath}
\usepackage{amsthm}
\usepackage{epstopdf}
\usepackage{hyperref}
\usepackage{listings}
\usepackage{subfiles}
\usepackage[utf8]{inputenc}
\usepackage{float}
\usepackage{tikz}
\usepackage{graphicx}
\usepackage{caption}
\usepackage{wrapfig}

\newcommand{\PRIMES}{\textsf{PRIMES}}
\newcommand{\cP}{\textsf{P}}
\newcommand{\coRP}{\textsf{coRP}}

\begin{document}

\section*{Exercise 36}
Known from the book
\begin{enumerate}
  \item $\PRIMES \in \cP$
  \item $\PRIMES \in \coRP$
\end{enumerate}
This means that there exists an efficient deterministic algorithm that recognizes $\PRIMES$ in
in $O((\log n)^{c \log \log \log n})$, for some $c\in \mathbb{N}$. This may take to much time. As mentioned
above the problem $\PRIMES$ is part of the class $\coRP$. So there is an
algorithm that if the input is $n$ and $n$ is a prime number it will recognize
$n$ as a prime. But it there is a probability of $\frac{1}{3}$ that the algorithm will output
that $n$ is a prime number even if $n$ is no prime number. With these
informations the friend of mine has to decide. If the calculation of $\PRIMES$
is part of a security procedure I would recommend to use the deterministic
algorithm. 


\end{document}

