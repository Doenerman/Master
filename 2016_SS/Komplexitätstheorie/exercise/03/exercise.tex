\documentclass[10pt]{article}
\usepackage{geometry}                % See geometry.pdf to learn the layout options. There are lots.
\usepackage{blindtext}
\usepackage[parfill]{parskip}    % Activate to begin paragraphs with an empty line rather than an indent
\usepackage{tikz}
\usetikzlibrary{arrows,automata,shadows,positioning,shapes}
\usepackage{graphicx}
\usepackage{amssymb}
\usepackage{amsmath}
\usepackage{amsthm}
\usepackage{epstopdf}
\usepackage{hyperref}
\usepackage{listings}
\usepackage{subfiles}
\usepackage[utf8]{inputenc}
\usepackage{float}
\usepackage{tikz}
\usepackage{graphicx}
\usepackage{caption}
\usepackage{wrapfig}

\newcommand{\nice}[1]{\textbf{#1}}

\begin{document}

\section*{Exercise 09}
  \subsection*{$\nice{P}\subsetneq\nice{E}$}
    \begin{proof}
      It is obvious that $\nice{P}\subseteq\nice{E}$ holds. Now construct a
      language $D\in \nice{E}$, but $D\not\in\nice{P}$.
      \begin{displaymath}
        D=\{\alpha \mid \text{$M_{\alpha}$ outputs $O$ after $\alpha$ steps}\}
      \end{displaymath}
      This is obviously decidable in \nice{E} since $M_{\alpha}$ can be
      simulated for each \textcolor{red}{hier fehlt noch was}

      Towards a contradiction assume there is a machine $M_{x}$ that decides
      $D$ in l$n^c, c\in\mathbb{N}$. Choose a $x^\prime>x^c$ such that
      $M_{x^\prime}=M_{x}$ this is possible since each TM can be represented by
      infinitely many strings.
      Now construct a TM $M_{y}$ that cleans the output, writes $x$ on the onto
      the output tape and behave like $M$ afterwards, $y>x$.
    \end{proof}


\section*{Exercise 10}
  \begin{math}
    \begin{array}{l | c l}
      \text{running times} & & \text{algorithm} \\ \hline
      & & \\
      & & \\
      \log \log n & & \text{For }o\leq i < \log \log n \\
      & & \\
      \log n      & & \hspace{5mm}\text{For }0\leq j < \log n \\
      & & \\
      T(j)=T(\log n)
                  & & \hspace{10mm} h=H(j) \\
      & & \\
      2^{\log n} = n & & \hspace{10mm}\text{For $x\in\{0,1\}^\star$ with $|x|=j$} \\
      & & \\
      (i|x|^i)^{1,5} < \psi < n^3 & &
                    \text{\hspace{15mm}Run U on $i$ for $i|x|^i$ steps. Save
                    output to $o$} \\
      & & \\
      H(|x|)\cdot 2^{|x|}=n
                & & \text{\hspace{15mm}Run $M_{\nice{SAT}_{H}}$ on x save output to
                $o^\prime$} \\
      & \\
      O(1)   & & \text{\hspace{15mm}if $o==O^\prime$} \\
      & \\
      O(1)      &\hspace{10mm} & \text{\hspace{20mm}return $i$}
    \end{array}
  \end{math}

  Let $T(n)$ be the running to of $H(n)$. Proof by induction.
  
  
  \begin{tabular}{ l l }
    Base Case: & \\
                     & $n\leq 2$ : The outer loop is never executed an thus
                        $T(n)\in O(1)\subseteq O(n^3)$. \\
    Induction Step: & \\
                    & \begin{math}
                        \begin{array}{l}
                          T(n) \in O(\log \log (n) \log (n) \left(
                          T(\log(n))+n(log(n)\cdot n + n)\right)\\
                          \in O\left( /log^2(n)T(\log(n))+n^2\cdot
                          \log^3(n)\right) \\
                          \in O\left( \log^2(n) \log \log^3(n) + n^3\cdot
                          \log^3(n)\right) \\
                          \in O\left( n^2\cdot \log^3(n)) \right) \\
                          \in O(n^3)
                        \end{array}
                      \end{math}
  \end{tabular}

  \section*{Exercise 11}
    Since $\nice{SAT}_{H}$ is \nice{NP}-complete, there is a reduction $f$ in polynomial
    time, say $O(n^{i})$ from \nice{SAT} to $\nice{SAT}_{H}$.

    We now construct a polynomial time algorithm $A$ that decides \nice{SAT} on
    $\varphi$ in $O(n^j)$, $j\geq i$:
    
    Let $N\in \mathbb{N}$ be the number such that $H(n)>i$ for $n>N$. 
    If $|\varphi| \leq N$ solve $\nice{SAT}(\varphi)$ using a brute force
    assignment. This can be done in constant time since the length of $\varphi$
    is bounded. 
    If $|\varphi|>N$, compute $\eta=f(\varphi)$. By construction of
    $f,\eta \in \nice{SAT}_{H} \Leftrightarrow \varphi \in \nice{SAT}$. If $\eta$
    is not of the form $\psi01^{n^{H(n)}}$ $\eta \neg\in \nice{SAT}_{H}$ and
    thus $\varphi\neg\in\nice{SAT}$, so return false. This check can be done in
    polynomial time since the length of $\varphi$ up to the marker can be
    counted in polynomial time an $H$ can be computed in polynomial time by
    definition.  

    Otherwise run $A(\psi)$ and forward its output. Proof by induction.

      \begin{tabular}{l l}
        Base Case: & \\
                  & As described above. \\
        Induction Step: & \\
                  & $f$ runs in polynomial time of $|\varphi|$ so $eta$ has to
                  be in $O(n^j)$. So $|\psi|<|\varphi|$ must be\\ &
                  sub linear in size of $|\varphi|$. Otherwise \\ & 
                  $|\psi|^{H(|\psi|)}\geq
                  |\varphi|^{H(|\varphi|)}>|\varphi|^j<|\varphi|^i$
                  and $f$ hasn't got enough time to print \\ &
                  sufficiently man $1$'s. On input $\psi$
                  $A$ runs at most $O(|\psi|)$ steps by the \\ &
                  induction step. The reduction can be performed in 
                  $O(|\varphi|^i)$ steps and \\ &
                  since $i\leq j$ $A$ terminates on $\varpi$ in
                  $O(|\psi|^j+|\varphi|^j)\in O(|\varphi|^j+|\varphi|^j) \in
                  O(|\varphi|^j)$
      \end{tabular}

\end{document}

